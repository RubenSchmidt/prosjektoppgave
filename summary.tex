% !TEX encoding = UTF-8 Unicode
%!TEX root = thesis.tex
% !TEX spellcheck = en-US
%%=========================================
\addcontentsline{toc}{section}{Abstract}
\section*{Abstract}
Old scanned zoning regulations are currently being vectorized using manual techniques in Norway. With the recent advances in image recognition and semantic segmentation using convolutional neural networks, there should be a possibility to use them for automatic vectorization of the regulations thus, lowering the cost and time spent doing manual work.

This paper gives an overview of the state-of-the-art regarding image segmentation using convolutional networks and vectorization techniques for automatic vectorization of raster maps. We also look at how the dataset needed to train a convolutional neural network can be automatically created by using already vectorized zoning regulations.

We show that the current techniques of vectorization need simple raster maps with uniform looks and clear polygons, qualities the zoning regulations do not have, and that the automatic creation of the dataset is not so simple as initially though because of the lack of information about the position of the vectorized polygons within the scanned zoning regulations.